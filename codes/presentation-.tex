\documentclass[aspectratio=43]{beamer}
\usepackage[english]{babel}
\usepackage{graphics}
\input{chapters/preamble}
\title{ASSIGNMENT 11} %->->->->-> Check hyperref title <-<-<-<-<-

\authoR{LAHARI} %You can change the Institution if you are from somewhere else
\date{JAN 5,2021}
%\logo{\includegraphics[width= 0.2\textwidth]{images/a-logo.png}}

\begin{document}
    
    \frame{\titlepage}
    
    \begin{frame}{QUESTION}
        In the circuit shown below, a positive edge-triggered D Flip-Flop is used for sampling input
data Din using clock CK. The XOR gate outputs 3.3 volts for logic HIGH and 0 volts for
logic LOW levels. The data bit and clock periods are equal and the value of ΔT/TCK = 0.15,
where the parameters ΔT and TCK are shown in the figure. Assume that the Flip-Flop and the
XOR gate are ideal.


    \end{frame}
\begin{frame}{diagram}
\includegraphics[width=1\textwidth]{20210105_184901.jpg}
If probability of input data bit(D input) transition in each clock period is 0.3, the average value ( in volts,accurate to two decimal places)
of voltage at node X,is o.8145V.

    
\end{frame}
    
\begin{frame}{diagram}
\begin{center}
\includegraphics[width=10cm,height=7cm,keepaspectratio]{td3.png}
\end{center}
    
\end{frame}
\begin{frame}{Explanation}
    volatage at X=$\frac{T-\Delta T}{T}.(probability).(voltage value)$\\\\\\
    X=1-$\frac{\Delta T}{T}$.(0.3).(3.3)\\\\
    X=(1-0.15).(0.3).(3.3)=0.8415V
\end{frame}
    
    
    
    
\end{document}
